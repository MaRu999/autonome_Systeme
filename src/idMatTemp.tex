%! Author = Matthias
%! Date = 15.10.2020

% Preamble
\documentclass[11pt]{article}

% Packages
\usepackage{amsmath}
\usepackage{german}

% Document
\begin{document}
    \subsection{Matrix Invertierung}\label{subsec:matrxinv}
    In diesem Kapitel folgt noch die genaue Beschreibung der Matrixinvertierung.
    Man augmentiert hierzu erst die Matrix mit der Identit"atsmatrix:
    \begin{equation}
        \left[\begin{array}{ccc|ccc}
                  \frac{\sqrt{2}}{2} & \frac{-\sqrt{2}}{2} & -L\sqrt{2} & 1 & 0 & 0 \\
                  \frac{\sqrt{2}}{2} & \frac{\sqrt{2}}{2} & -L\sqrt{2} & 0 & 1 & 0\\
                  \frac{\sqrt{2}}{2} & \frac{-\sqrt{2}}{2} & L\sqrt{2} & 0 & 0 & 1
        \end{array}\right]\label{eq:startinv}
    \end{equation}
    Jetzt muss man in der ersten Spalte alle Werte auf 0 bekommen, au"ser dem an Zeile 1, Spalte 1 (Pivot).
    Hierzu addiert man zuerst Reihe 1 zu Reihe 2:
    \begin{equation}
        \left[\begin{array}{ccc|ccc}
                  \frac{\sqrt{2}}{2r}  & -\frac{sqrt{2}}{2r} & -\frac{\sqrt{2}L}{r} & 1 & 0 & 0 \\
                  0                    & -\frac{sqrt{2}}{r}  & 0                    & 1 & 1 & 0 \\
                  -\frac{\sqrt{2}}{2r} & \frac{sqrt{2}}{2r}  & -\frac{\sqrt{2}L}{r} & 0 & 0 & 1
        \end{array}\right]\label{eq:equationr1addr3}
    \end{equation}
    Jetzt addiert man Reihe 1 zu Reihe 3:
    \begin{equation}
        \left[\begin{array}{ccc|ccc}
                  \frac{\sqrt{2}}{2r} & -\frac{sqrt{2}}{2r} & -\frac{\sqrt{2}L}{r}  & 1 & 0 & 0 \\
                  0                   & -\frac{sqrt{2}}{r}  & 0                     & 1 & 1 & 0 \\
                  0                   & 0                   & -\frac{2\sqrt{2}L}{r} & 1 & 0 & 1
        \end{array}\right]\label{eq:invr3tor1}
    \end{equation}
    Man multipliziert Reihe 1 mit $\sqrt{2}$:
    \begin{equation}
        \left[\begin{array}{ccc|ccc}
                  1 & -1                 & -2L                   & \sqrt{2}r & 0 & 0 \\
                  0 & -\frac{sqrt{2}}{r} & 0                     & 1         & 1 & 0 \\
                  0 & 0                  & -\frac{2\sqrt{2}L}{r} & 1         & 0 & 1
        \end{array}\right]\label{eq:invsqrtmult}
    \end{equation}
    Jetzt m"ussen in der zweiten Spalte alle Werte auf Null gesetzt werden, au"ser dem an Zeile 2, Spalte 2 (Pivot).
    Hierzu multipliziert Reihe 2 mit $-\frac{\sqrt{2}r}{2}$:
    \begin{equation}
        \left[\begin{array}{ccc|ccc}
                  1 & -1 & -2L                   & \sqrt{2}r            & 0                    & 0 \\
                  0 & 1  & 0                     & -\frac{\sqrt{2}r}{2} & -\frac{\sqrt{2}r}{2} & 0 \\
                  0 & 0  & -\frac{2\sqrt{2}L}{r} & 1                    & 0                    & 1
        \end{array}\right]\label{eq:pivottwo}
    \end{equation}
    Dann Reihe 2 zu Reihe 1 addieren:
    \begin{equation}
        \left[\begin{array}{ccc|ccc}
                  1 & 0 & -2L                   & \frac{\sqrt{2}r}{2}  & -\frac{\sqrt{2}r}{2} & 0 \\
                  0 & 1 & 0                     & -\frac{\sqrt{2}r}{2} & -\frac{\sqrt{2}r}{2} & 0 \\
                  0 & 0 & -\frac{2\sqrt{2}L}{r} & 1                    & 0                    & 1
        \end{array}\right]\label{eq:r2tor1}
    \end{equation}
    Nun werden in der dritten Spalte alle Werte auf Null gesetzt, au"ser dem an Zeile 3, Spalte 3 (Pivot).
    Dazu multipliziert Reihe 3 mit $-\frac{\sqrt{2}r}{4L}$:
    \begin{equation}
        \left[\begin{array}{ccc|ccc}
                  1 & 0 & -2L & \frac{\sqrt{2}r}{2}   & -\frac{\sqrt{2}r}{2} & 0                     \\
                  0 & 1 & 0   & -\frac{\sqrt{2}r}{2}  & -\frac{\sqrt{2}r}{2} & 0                     \\
                  0 & 0 & 1   & -\frac{\sqrt{2}r}{4L} & 0                    & -\frac{\sqrt{2}r}{4L}
        \end{array}\right]\label{eq:pivot3}
    \end{equation}
    Als N"achstes zieht man Reihe 3 mal $-2L$ von Reihe 1 ab:
    \begin{equation}
        \left[\begin{array}{ccc|ccc}
                  1 & 0 & -0 & 0                     & -\frac{\sqrt{2}r}{2} & -\frac{\sqrt{2}r}{2}  \\
                  0 & 1 & 0  & -\frac{\sqrt{2}r}{2}  & -\frac{\sqrt{2}r}{2} & 0                     \\
                  0 & 0 & 1  & -\frac{\sqrt{2}r}{4L} & 0                    & -\frac{\sqrt{2}r}{4L}
        \end{array}\right]\label{eq:mult3p1}
    \end{equation}
    Da nun die Identit"atsmatrix links steht, haben wir somit die Inverse gefunden:
    \begin{equation}
        \prescript{3}{}{J}^{-1} =
        \begin{bmatrix}
            0 & -\frac{r}{\sqrt{2}} & -\frac{r}{\sqrt{2}} \\
            -\frac{r}{\sqrt{2}} & -\frac{r}{\sqrt{2}} & 0 \\
            -\frac{r}{2\sqrt{2}L} & 0 & -\frac{r}{2\sqrt{2}L} \\
        \end{bmatrix}\label{eq:finallydone}
    \end{equation}


\end{document}