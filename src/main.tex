%! Author = Matthias
%! Date = 01.10.2020

% Preamble
\documentclass[11pt]{article}

% Packages
\usepackage{amsmath}

% Document
\begin{document}
    \section{Autonome Systeme - Bewegung - V 2}\label{sec:vtwo}
    Freie Vektoren -> wenn Position keine Rolle spielt (e.g. Geschwindigkeit) -> Vektor mit 0 statt 1 ergaenzen -> mit Transformationsmatrix multiplizieren -> Rotation ohne Verschiebevektor\newline
    1. Kraefte, Geschwindigkeiten, Momente (freie Vektoren) -> vom Bezugssystem in anderes transformieren -> was beachten?
    Verschiebevektor wird nicht benoetig (freie Vektoren haengen nicht von Position ab) -> nur Rotationsmatrix noetig, dritte Zeile mit 0 erweitern
    2. Wie funktioniert die Transformation von Vektoren im 3D-Raum?
    3. Wie sieht die Transformationsmatrix in 3D aus?
    4. Wie wird die Pose eines Roboters in 2D als Vektor beschrieben?
    - Referenzsystem -> Koordinatensystem U; Roboter bekommt eigenes, lokales Koordinatensystem R;
    - Position des Roboters = Position des Ursprungs von R in U (Koordinaten x y);
    - Orientierung des Roboters = Verdrehung zwischen R und U (Winkel zwischen  x-Achse von R und der x-Achse von U) = Winkel zeta
    - Pose des Roboters = Kombination von Position und Orientierung
    5. Wie sieht der Geschwindigkeitsvektor eines Roboters in 2D aus?
    6. Wie wird die Geschwindigkeit eines Roboters in 2D transformiert?
    7. Was versteht man unter der Vorwaertskinematik?
    8. Leiten sie die Vorwaertskinematik eines Roboters mit Differentialantrieb aus den geometrischen Verhaeltnissen her
    9. Welche Annahmen werden bei der Ableitung der kinematischen Zwangsbedingungen fuer Raeder getroffen?
    10. Welche beiden Bewegungsrichtungen werden fuer jedes Rad bei der Ableitung der Zwangsbedingungen betrachtet?
    11. Leiten sie die Zwangsbedingungen fuer das Standardrad her.
    12. Wodurch unterscheiden sich die Zwangsbedingungen des gelenkten vom ungelenkten Standardrad?
    13. Erklaeren sie die Ableitung der Zwangsbedingung fuer das Mecanum-Wheel.
    14. Wodurch unterscheiden sich die Zwangsbedingungen des Kugelrades vom ungelenkten Standardrad?

\end{document}